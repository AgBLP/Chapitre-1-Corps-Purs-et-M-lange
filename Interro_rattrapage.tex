\modeCorrection

\nomPrenomClasse

\begin{center}
\begin{Large}
    
    Interrogation de cours : Corps purs et mélanges au quotidien (10min)
\end{Large}
\end{center}
\vspace{1cm}


\question{Citer un test chimique pour reconnaitre la présence de dioxyde de carbone \chemform{CO_2}.}{Le \chemform{CO_2} réagit avec l'eaux de chaux pour former un précipité blanc. L'eau de chaux se trouble donc en présence de \chemform{CO_2}.}{2}

\question{L'acétone est miscible avec l'eau. Que cela veut-il dire ? Citer deux espèces chimiques non miscibles.}{Le mélange eau-acétone est homogène. L'huile et l'eau ne sont pas miscibles. On pourrait également citer le graphite avec de l'huile par exemple.}{2}

\question{Donner la formule de la densité d'une espèce chimique. Donner la signification et l'unité de chaque grandeur physique introduite (y compris la densité).}{La densité $d$ d'une espèce chimique est donnée par la formule :
\begin{equation*}
    d = \frac{\rho_{\text{espèce}}}{\rho_{eau}}
\end{equation*} avec $\rho$ la masse volumique (en g.cm$^{-3}$ ou g.L$^{-1}$). La densité s'exprime sans unité.}{3}
\\
\newline
\newline